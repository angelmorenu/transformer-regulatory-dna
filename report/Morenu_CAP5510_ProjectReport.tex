\documentclass[conference]{IEEEtran}
\IEEEoverridecommandlockouts
\usepackage{hyperref}
\usepackage{graphicx}
\usepackage{booktabs}
\usepackage{amsmath}
\usepackage{cite}
\usepackage{parskip}

\title{Transformer Fine-Tuning for Regulatory DNA:\\
Classifying Functional Elements and Scoring Variant Effects}

\author{
  \IEEEauthorblockN{Angel Morenu}
  \IEEEauthorblockA{M.S. Applied Data Science, University of Florida\\
  Email: angel.morenu@ufl.edu}
}

\begin{document}
\maketitle

\begin{abstract}
We evaluate transformer-based language models for DNA on two tasks: (i) functional element classification (promoters, enhancers, DNase, TF binding, histone marks) and (ii) variant effect prediction via in-silico mutagenesis. We fine-tune DNABERT-2 and Nucleotide Transformer and compare them with CNN baselines (Basset/Basenji) and linear probes on frozen embeddings. Primary metrics are AUROC and PR-AUC; secondary metrics include runtime, GPU memory, and cross-cell-type transfer. This work extends CAP 5510 Module 2 (sequence similarity/alignment) by applying NLP-style models to DNA sequences.
\end{abstract}

\section{Introduction}
Briefly motivate the problem and related work. Focus on your \textbf{contributions and observations}.

\section{Datasets}
Describe ENCODE / Roadmap / DeepSEA sources, label construction, and splits.

\section{Methods}
\subsection{Baselines}
Summarize Basset/Basenji setup.
\subsection{Transformers}
DNABERT-2 and Nucleotide Transformer, tokenization, context length.
\subsection{Training Details}
Batch size, learning rate, epochs, hardware.

\section{Experiments}
\subsection{Classification}
Task definitions and evaluation protocol.
\subsection{Variant Effect Prediction}
In-silico mutagenesis; scoring strategy.

\section{Results}
\subsection{Accuracy Metrics}
Include AUROC/PR-AUC tables and confidence intervals.
\subsection{Efficiency}
Runtime and GPU memory plots.
\subsection{Transfer}
Cross-cell-type generalization.

\section{Discussion}
Interpret results; when/why do transformers help? Limitations.

\section{Lessons Learned \& Future Work}
Reflection and next steps.

\bibliographystyle{IEEEtran}
\bibliography{refs}
\end{document}